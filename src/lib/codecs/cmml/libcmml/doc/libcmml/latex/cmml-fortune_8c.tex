\section{cmml-fortune.c File Reference}
\label{cmml-fortune_8c}\index{cmml-fortune.c@{cmml-fortune.c}}
{\tt \#include \char`\"{}config.h\char`\"{}}\par
{\tt \#include $<$stdio.h$>$}\par
{\tt \#include $<$stdlib.h$>$}\par
{\tt \#include $<$fcntl.h$>$}\par
{\tt \#include $<$ctype.h$>$}\par
{\tt \#include $<$string.h$>$}\par
{\tt \#include $<$unistd.h$>$}\par
{\tt \#include $<$cmml.h$>$}\par
\subsection*{Defines}
\begin{CompactItemize}
\item 
\#define {\bf DEFAULT\_\-DURATION}\ \char`\"{}00:30\char`\"{}
\item 
\#define {\bf DEFAULT\_\-ENCODING}\ \char`\"{}UTF-8\char`\"{}
\item 
\#define {\bf DEFAULT\_\-SHORT\_\-COMMAND}\ \char`\"{}fortune -s -n 80\char`\"{}
\item 
\#define {\bf DEFAULT\_\-LONG\_\-COMMAND}\ \char`\"{}fortune -l\char`\"{}
\item 
\#define {\bf BUFSIZE}\ 10000
\end{CompactItemize}
\subsection*{Functions}
\begin{CompactItemize}
\item 
void {\bf Print\-Usage} (char $\ast$prog)
\item 
char $\ast$ {\bf get\_\-fortune} (char $\ast$command)
\item 
int {\bf main} (int argc, char $\ast$argv[$\,$])
\end{CompactItemize}


\subsection{Detailed Description}
cmml-fortune: generates a valid CMML document with random content.



\footnotesize\begin{verbatim}
Usage: cmml-fortune [options]
Generate a random CMML file using an external program
to provide text (fortune cookies by default).
 
Possible options:
  -d timespec, --duration timespec
                 Specify the duration of the generated CMML file
                 ('00:30' by default)
  -e encoding, --encoding encoding
                 Specify the encoding of the generated CMML file
                 ('UTF-8' by default)
 
  -o filename, --output filename
                 Specify the output filename. The file is written
                 to standard output by default.
 
  -s command,  --short-command command
                 Specify the command to use to generate short text
                 for the title and anchors ('fortune -s -n 80' by default)
  -l command,  --long-command command
                 Specify the command to use to generate long text
                 for descriptions ('fortune -l' by default)
  -h, --help     Display this help and exit
  -v, --version  Display version information and exit
\end{verbatim}
\normalsize


\subsection{Define Documentation}
\index{cmml-fortune.c@{cmml-fortune.c}!DEFAULT_DURATION@{DEFAULT\_\-DURATION}}
\index{DEFAULT_DURATION@{DEFAULT\_\-DURATION}!cmml-fortune.c@{cmml-fortune.c}}
\subsubsection{\setlength{\rightskip}{0pt plus 5cm}\#define DEFAULT\_\-DURATION\ \char`\"{}00:30\char`\"{}}\label{cmml-fortune_8c_a0}


DEFAULT\_\-DURATION: default duration of created CMML file \index{cmml-fortune.c@{cmml-fortune.c}!DEFAULT_ENCODING@{DEFAULT\_\-ENCODING}}
\index{DEFAULT_ENCODING@{DEFAULT\_\-ENCODING}!cmml-fortune.c@{cmml-fortune.c}}
\subsubsection{\setlength{\rightskip}{0pt plus 5cm}\#define DEFAULT\_\-ENCODING\ \char`\"{}UTF-8\char`\"{}}\label{cmml-fortune_8c_a1}


DEFAULT\_\-ENCODING: default encoding format of created CMML file \index{cmml-fortune.c@{cmml-fortune.c}!DEFAULT_SHORT_COMMAND@{DEFAULT\_\-SHORT\_\-COMMAND}}
\index{DEFAULT_SHORT_COMMAND@{DEFAULT\_\-SHORT\_\-COMMAND}!cmml-fortune.c@{cmml-fortune.c}}
\subsubsection{\setlength{\rightskip}{0pt plus 5cm}\#define DEFAULT\_\-SHORT\_\-COMMAND\ \char`\"{}fortune -s -n 80\char`\"{}}\label{cmml-fortune_8c_a2}


DEFAULT\_\-SHORT\_\-COMMAND: default command to create short descriptions \index{cmml-fortune.c@{cmml-fortune.c}!DEFAULT_LONG_COMMAND@{DEFAULT\_\-LONG\_\-COMMAND}}
\index{DEFAULT_LONG_COMMAND@{DEFAULT\_\-LONG\_\-COMMAND}!cmml-fortune.c@{cmml-fortune.c}}
\subsubsection{\setlength{\rightskip}{0pt plus 5cm}\#define DEFAULT\_\-LONG\_\-COMMAND\ \char`\"{}fortune -l\char`\"{}}\label{cmml-fortune_8c_a3}


DEFAULT\_\-LONG\_\-COMMAND: default command to create long descriptions \index{cmml-fortune.c@{cmml-fortune.c}!BUFSIZE@{BUFSIZE}}
\index{BUFSIZE@{BUFSIZE}!cmml-fortune.c@{cmml-fortune.c}}
\subsubsection{\setlength{\rightskip}{0pt plus 5cm}\#define BUFSIZE\ 10000}\label{cmml-fortune_8c_a4}


BUFSIZE: default buffer size for printing 

\subsection{Function Documentation}
\index{cmml-fortune.c@{cmml-fortune.c}!PrintUsage@{PrintUsage}}
\index{PrintUsage@{PrintUsage}!cmml-fortune.c@{cmml-fortune.c}}
\subsubsection{\setlength{\rightskip}{0pt plus 5cm}void Print\-Usage (char $\ast$ {\em prog})\hspace{0.3cm}{\tt  [static]}}\label{cmml-fortune_8c_a5}


Print\-Usage: prints out help on how to use this program

\begin{Desc}
\item[Parameters:]
\begin{description}
\item[{\em prog}]the program's name \end{description}
\end{Desc}
\index{cmml-fortune.c@{cmml-fortune.c}!get_fortune@{get\_\-fortune}}
\index{get_fortune@{get\_\-fortune}!cmml-fortune.c@{cmml-fortune.c}}
\subsubsection{\setlength{\rightskip}{0pt plus 5cm}char$\ast$ get\_\-fortune (char $\ast$ {\em command})\hspace{0.3cm}{\tt  [static]}}\label{cmml-fortune_8c_a6}


get\_\-fortune: generates random text using the given command

\begin{Desc}
\item[Parameters:]
\begin{description}
\item[{\em command}]the command to use to generate text\end{description}
\end{Desc}
\begin{Desc}
\item[Returns:]the generated text \end{Desc}
\index{cmml-fortune.c@{cmml-fortune.c}!main@{main}}
\index{main@{main}!cmml-fortune.c@{cmml-fortune.c}}
\subsubsection{\setlength{\rightskip}{0pt plus 5cm}int main (int {\em argc}, char $\ast$ {\em argv}[$\,$])}\label{cmml-fortune_8c_a7}


main function of cmml-fortune, which opens the CMML file, seeks to any given offsets, registers the callbacks, and then steps through the file in chunks of BUFSIZE size, during which the callbacks get activated as the relevant elements get parsed. 